\documentclass{book}

\begin{document}

\chapter{Disjoint set}

Hlavným cieľom tejto kapitoly je vysvetliť, čo to disjoint set je, kde je možné
túto datovú štruktúru použiť a akými spôsobmi ju možno implementovať.

\section{Úvod}

Set je datová štruktúra, ktorá združuje množinu vzájomne rozdielnych
(disjunktných) dát. Rozdiel medzi setom a Disjoint setom (taktiež nazývaná
Union-Find alebo Merge-find set) je ten, že Disjoint set rozdeľuje tieto prvky
do podmnožin. Na počiatku máme množinu všetkých prvkov, ktoré sa nachádzajú v
Disjoint sete a každý tento prvok tvorí samostatnú podmnožinu. Táto datová
štruktúra definuje 3 operácie:

\begin{itemize}
	\item \textbf{MakeSet ( x )} - pridanie prvku \textit{x} do množiny a
		jeho označenie ako samostatnej podmnožiny, 
	\item \textbf{Find ( x )} - zistí identifikátor množiny, v ktorej sa
		prvok \textit{x} nachádza, 
	\item \textbf{Union ( x, y )} - zjednotenie množin, ktoré obsahujú
		prvky \textit{x} a \textit{y}. 
\end{itemize}

Jednotlivé operácie tejto datovej štruktury možno implementovať rôznym spôsobom
-- o týchto spôsoboch implementácie je možné sa dočítať nižšie.

\section{Implementácia datovej štruktury}

Táto datová štruktúra je implementovaná ako les, kde každý strom reprezentuje
jednu podmnožinu prvkov. Tieto stromy sa následne môžu zjednocovať a tvoriť tak
novú podmnožinu. \newline \newline 

Čo je to teda ten les a strom? \newline \newline

Pre definovanie lesu je potrebné najprv definovať čo je to strom, graf a ďaľšie
pojmy z teórie grafov:

\begin{itemize}
	\item \textbf{(Neorientovaný) graf} - je usporiadaná dvojica $(V, E)$,
	kde:
		\begin{itemize}
			\item \textit{V} - je neprázdna množina vrcholov,
			\item \textit{E} - je množina hran (neusporiadaná
				dvojica vrcholov, značíme ju ${u, v}$),
		\end{itemize}
	\item \textbf{Sled} - je sekvencia $v_{0}, e_{1}, v_{1}, e_{2}, v_{2},
		..., e_{k}, v_{k}$, kde $e_{i} =\{v_{i-1}, v_{i}\}$ a $e_{i}
		\in E(G), \forall i \in {1, ..., k}$.  
	\item \textbf{Cesta} - je sled, v ktorom sa neopakujú vrcholy.
	\item \textbf{Súvislý graf} - Graf $G$ nazveme súvislým práve vtedy,
		keď pre každé 2 vrcholy $u$ a $v$ existuje $u-v-cesta$, 
	\item \textbf{Kružnica} - Majme $n>=3$, Kružnicou dĺžky $n$ (kružnica
		pozostávajúca z $n$ vrcholov) je graf $(\{1,...,n\}, \{\{i, i +
		1\} | i \in \{1, ..., n - 1\}\} \cup \{\{1, n\}\})$,
	\item \textbf{Podgraf} - Graf $H$ je podgrafom grafu $G$ práve vtedy,
		keď $V(H) \subseteq V(G)$ 
	\item \textbf{Acyklický graf} - Graf nazveme acyklický práve vtedy, keď
		neobsahuje ako podgraf kružnicu.
	\item \textbf{Strom} - Graf $G$ nazveme stromom práve vtedy, keď je
		súvislý a zároveň acyklický,
	\item \textbf{Les} - Graf $G$ nazveme lesom práve vtedy, keď je
		acyklický.
\end{itemize}

\section{Použitie}

Keďže už vieme, čo to ten Disjoint set je, bolo by dobré vedieť k čomu nám
takáto datová štruktúra je a kde všade je možné ju použiť. Možné použitia:
\begin{itemize}
	\item Udržovanie informácií o súvislých komponentách,
	\item Hľadanie minimálnej kostry na základe Kruskalovho algoritmu, 
	\item Detekcia cyklu.
\end{itemize}

\end{document}
