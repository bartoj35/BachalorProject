\begin{introduction}

V~dnešnej dobe veľká časť programátorov implementuje v~rámci svojej práce riešenia na konkrétne problémy, ale nie dátové štruktúry pre to potrebné. V~moment keď narazia na potrebu použitia dátovej štruktúry využívajú v~minulosti použité implementácie alebo si implementáciu nájdu online (napríklad na stránkach ako je github alebo stackoverflow). 

Sú však tieto implementacie správne? Môžeme sa naozaj spoľahnúť, že neobsahujú žiadne chyby, ktoré by mohli viesť k~nesprávnym výsledkom alebo v~horšom prípade eskalovať k~väčším problémom? Na tieto stránky predsa môže prispieť ktokoľvek. Kód, ktorý možno na týchto stránkach nájsť nie je nijako regulovaný, jediný spôsob pre zistenie či sa týka o~dobrý zdroj je popularita autora (počet ľudí sledujúcich užívateľa na githube alebo reputácia v~prípade stackoverflow). Aby sme však zaistili, že kód spĺňa naše požiadavky je potreba ho poriadne testovať. 

Testovanie je možné vykonať napriklad pomocou jednotkových testov (tzv. unit testov) alebo testom komponenty. Prípadne možno zvoliť nástroje typu symbolic engine \ref{symbolic_engine} -- ten nám umožní otestovať volania funkcií za pomoci symbolických premenných, ktoré menia svoju hodnotu, aby umožnili vykonať všetky možné priechody funkciou. Tento proces môže byť z~dôvodou častého vetvenia časovo a~výpočetne veľmi náročný a~tiež objaví skôr chyby poškodzujúce pamäť -- chyby vznikajúce nesprávnou prácou s~pamäťou (napríklad prístup mimo pole, či už na zásobníku, alebo na halde alebo dereferencia pamäti, ku~ktorej by sme nemali mať prístup). 

Existuje však nejaké vhodnejšie riešenie? Našťastie odpoveď je áno. Tomuto riešeniu sa hovorí verifikácia kódu. Jedná sa o~proces, ktorý nám umožňuje formálne dokázať, že náš kód funguje správne a~vykonáva práve to, čo od neho očakávame a~to za pomoci software a presne špecifikovaných informácií ako by mali vyzerať parametry funkcie, ako by mal vyzerať výsledok funkcie atď. Verifikáciu možno vykonať pomocou rôznych nástrojov a~frameworkov, príkladom pre jazyky C je framework Frama-C \ref{frama-c_uvod}.

\section{Cieľ práce}

V~rámci tejto práce sa oboznámime s~dátovou štruktúrou Union-Find \ref{union-find_uvod}, zistíme aké sú možnosti implementácie tejto štruktúry pre dosiahnutie zrýchlenia. V~programovacom jazyku C vytvoríme implementáciu všetkých týchto spôsobov. Zoznámime sa s~frameworkom Frama-C \ref{frama-c_uvod}, ktorý slúži pre verifikáciu programov v~programovacom jazyku C, následne ho využijeme pre verifikáciu nami vytvorených implementácií dátovej štruktúry. Overíme správnosť našej implementácie testovaním \ref{testovanie} a~tiež sa pozrieme na výkonnostné rozdiely naprieč implementáciami \ref{benchmark}.

\end{introduction}