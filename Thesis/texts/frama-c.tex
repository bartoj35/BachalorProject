\chapter{Frama-C} 

V~tejto kapitole sa budeme venovať frameworku slúžiacemu pre verifikáciu zdrojových kódov v~programovacom jazyku C. Tým frameworkom je \emph{Frama-C}.

Vysvetlíme si, ako tento framework funguje, popíšeme si, čo je jeho základným stavebným kameňom, rozoberieme si význam a~využitie vybraných kľúčových slov jazyku \emph{ACSL}\footnote{ANSI/ISO C Specification Language} (jazyk používaný v~rámci frameworku) a~na záver sa pozrieme na pluginy, ktoré sme pri práci s frameworkom použili.

\section{Čo to je Frama-C?} \label{frama-c_uvod}

\emph{Frama-C} je open-source\footnote{zdrojové kódy sú verejné prístupné a~ktokoľvek ich môže modifikovať pri dodržaní licenčných práv} framework slúžiaci pre analýzu zdrojových kódov. Jeho použitím sa užívateľ snaží odhaliť neočakávaného chovania a~preukázať správnu funkčnosť zdrojového kódu alebo implementácie. Framework užívateľovi umožňuje definovať požiadavky na chovanie funkcionality a~pomocou nich ukázať, že jeho implementácia tieto požiadavky splňuje. Pri vhodnej tvorbe požiadavkov preto možno zaručiť, že funkcionalita neobsahuje chyby.\cite{frama-c-uvod}

Framework tiež podporuje pluginy\footnote{rozšírenia pridávajúce rôzne funkcionality}, ktoré môže byť často veľmi prospešné pri overovaní funkcionality programu. My sme sa rozhodli v rámci našej práce použiť niekoľko z~nich, konkrétne \emph{WP}, \emph{EVA} a \emph{RTE}. Tieto pluginy si bližšie popíšeme v~kapitole \ref{chapter:pluginy}.

\section{Kontrakt}

Framework pre špecifikáciu požiadavkov na funkcionality používa takzvaný kontrakt -- jedná sa o~základnú stavebnú jednotku tohto frameworku.

Kontrakt pozostává z~požiadavkov, ktoré by mali platiť pre jednotlivé funkcie, cykly či vetvenia a~nachádzajú sa v~anotačných komentároch. Tieto požiadavky sú popísané pomocou jazyku \emph{ACSL}. Ten obsahuje vstavané funkcie a~predikáty, datové typy premenných, takzvaný ghost code a~ghost premenné. \cite{obsah-acsl} Kontrakt sa bežne skladá zo~vstupných a~výstupných požiadavkov, pre ktoré sa používajú kľúčové slová \emph{requires} a~\emph{ensures}. Tieto kľúčové slová a~niekoľko ďaľších komponent tohto jazyku si vysvetlíme v~nasledujúcej časti tejto kapitoly.

\begin{figure}[H]
    \centering
    \captionsetup{justification=centering}
\begin{lstlisting}[language=C]
//@ requires <some_condition>;
void print ( int number );
\end{lstlisting}
    \caption{Ukážka jednoriadkového kontraktu}
    \label{fig:jednoduchy-kontrakt-jednoriadkovy}
\end{figure}

\begin{figure}[H]
    \centering
    \captionsetup{justification=centering}
\begin{lstlisting}[language=C]
/*@
  @ requires <some_condition>;
  @
  @ ensures <some_condition>;
*/
void append ( char ** destination, char * source );
\end{lstlisting}
    \caption{Ukážka viacriadkového kontraktu}
    \label{fig:jednoduchy-kontrakt-viacriadkovy}
\end{figure}

\section{Kľúčové slova jazyku ACSL}

Anotačný jazyk \emph{ACSL} slúži pre špecifikáciu požiadavkov na funkcie, cykly a vetvenia. Kľúčové slová môžu slúžiť pre označenie vstupný alebo výstupných podmienok, logických výrazov (v rámci nich možno použiť indikátory reprezentujúce hodnoty pravdy a~nepravdy vo~forme kľúčových slov \emph{\textbackslash true} a \emph{\textbackslash false}) a~funkcií, označenie kódu prístupného len v~rámci kontraktov alebo aj identifikátory stavu premenných (napríklad stav parametru pred a~po vykonaní funkcie).

Kľúčové slova jazyku \emph{ACSL} delíme na termy, predikáty a~zvyšné kľúčové slová. Termy a~predikáty sú špecifické tým, že na rozdiel od zvyšku sa ich názov začína spätným lomítkom (napríklad už zmienené \emph{\textbackslash true} a \emph{\textbackslash false}). \cite{keywords-acsl}

\subsection{requires} \label{chapter:requires}

Klauzula \emph{requires} sa využíva v~rámci kontraktu funkcie a~špecifikuje vstupné podmienky funkcie (napríklad aké musia byť hodnoty parametrov a~aká musí byť štruktúra ich dát). Pre pokračovanie v analýze požiadavkov na funkciu je potrebné, aby klauzula \emph{requires} bola splnená (ak vstupné dáta nesplňujú požiadavky, ktoré sme po nich požadovali nemôžme garantovať ako bude výsledok vyzerať). Zároveň v prípade väčšieho množstva požiadavkov je možné ich reťazenie pomocou operátoru logického súčinu (\&\&) alebo viackrát využiť túto klauzulu.

Použitie klauzule \emph{requires} si ukážeme na funkcií \emph{logarithm}, ktorá vypočíta hodnotu dekadického logaritmu pre hodnotu nachádzajúcu v~argumente. Nakoľko je logaritmus definovaný iba pre nezáporné hodnoty, požadujeme nezáporný argument aj v~rámci našej funkcie (viď kontrakt v ukážke \ref{fig:kontrakt-requires}).

\begin{figure}[H]
    \centering
    \captionsetup{justification=centering}
\begin{lstlisting}[language=C]
//@ requires number >= 0;
int logarithm ( int number );
\end{lstlisting}
    \caption{Príklad klauzule \emph{requires}}
    \label{fig:kontrakt-requires}
\end{figure}

Ak kontrakt neobsahuje túto klauzulu predpokladá sa, že vstupné požiadavky nie su žiadne.\cite{default-requires-acsl}. To si ukážeme na funkcií, ktorá na štandardný výstup vypíše číslo jeden (viď ukážka \ref{fig:kontrakt-bez-requires}).

\begin{figure}[H]
    \centering
    \captionsetup{justification=centering}
\begin{lstlisting}[language=C]
//@ requires \true;
void printOne ( void );
\end{lstlisting}
    \caption{Ekvivalent kontraktu bez klauzule \emph{requires}}
    \label{fig:kontrakt-bez-requires}
\end{figure}

\subsection{ensures} \label{chapter:ensures}

Klauzula \emph{ensures} dopĺňa v rámci kontraktu funkcie klauzulu \emph{requires}. Klauzula \emph{ensures} špecifikuje výstupné podmienky funkcie (napríklad aké podmienky musí splňovať návratová hodnota a~čo musí platiť pre vstupné parametry, \textbf{ale až na konci funkcie} -- napríklad v prípade zmeny hodnôt v~poli, ktoré bolo predané funkcií ako argument). Pre špecifikáciu viacerých výstupných podmienok možno využiť operátor logického súčinu (\&\&) alebo viackrát použiť túto klauzulu.

Použitie klauzule \emph{ensures} si ukážeme na funkcií \emph{square}, ktorá zmení hodnotu argumentu na hodnotu jej druhú mocniny. Keďže druhá mocnina ľubovolného čísla je číslo nezáporné, budeme požadovať nezápornú hodnotu argumentu na konci funkcie -- to dosiahneme pomocou klauzule \emph{ensures} (viď kontrakt v ukážke \ref{fig:kontrakt-ensures}).

\begin{figure}[H]
    \centering
    \captionsetup{justification=centering}
\begin{lstlisting}[language=C]
//@ ensures ( * number ) >= 0;
void square ( int * number );
\end{lstlisting}
    \caption{Ukážka klauzule \emph{requires}}
    \label{fig:kontrakt-ensures}
\end{figure}

Podobne ako v~prípade klauzule \emph{requires} sa pri vynechaní klauzule \emph{ensures} predpokladá, že na funkciu nie su kladené žiadne výstupné podmienky. Teda je táto možnosť ekvivalentná následujúcemu zápisu:

\begin{figure}[H]
    \centering
    \captionsetup{justification=centering}
\begin{lstlisting}[language=C]
//@ ensures \true;
void printOne ( void );
\end{lstlisting}
    \caption{Ekvivalent kontraktu bez klauzule \emph{ensures}}
    \label{fig:kontrakt-bez-ensures}
\end{figure}

Táto klauzula zároveň obsahuje jednu limitáciu a~to konkrétne v~prípade použitia kľúčového slova \textbf{goto}. Pri použití tohto kľúčového slova nie je klauzula \emph{ensures} overovaná. \cite{ensures-goto-acsl}

\subsection{assigns}

Toto kľúčové slovo sa využíva v rámci kontraktu funkcie a~špecifikuje, ktoré pamäťové bunky (premenné) sa môžu v~rámci funkcie meniť. V~prípade viacerých pamäťových buniek je možné využiť viacero klauzulí alebo ich vypísať ako zoznam viacerých premenných (v~takom prípade musia byť pamäťové bunky oddelené čiarkou). Pamäťové bunky, ktoré sa v~rámci tejto klauzule špecifikujú musia byť argumentom funkcie alebo globálne premenné \footnote{nemôže sa jednať o~lokálne premenné v~rámci funkcie nakoľko tieto premenné na začiatku funkcie ešte neexistujú}.

Použitie tejto klauzule si ukážeme na funkcií \emph{swap}, ktorá bude vykonávať výmenu obsahu 2 premenných typu int. Nakoľko funkcia zmení obsah pamäťových buniek vstupujúcich do funkcie je vhodné tieto zmeny v~rámci funkcie označiť a~to práve pomocou klauzule \emph{assings}.

\begin{figure}[H]
    \centering
    \captionsetup{justification=centering}
\begin{lstlisting}[language=C]
//@ assigns * number1, * number2;
void swap ( int * number1, int * number2 );
\end{lstlisting}
    \caption{Ukážka klauzule \emph{assings}}
    \label{fig:kontrakt-assigns}
\end{figure}

Nakoľko niektoré funkcie nemusia do takýchto pamäťových buniek zapisovať je možno túto klauzulu vynechať. Prípadne možno pre úplnosť špecifikovať, že sa nezapisuje do žiadnej takejto bunky následovne:

\begin{figure}[H]
    \centering
    \captionsetup{justification=centering}
\begin{lstlisting}[language=C]
//@ assigns \nothing;
void printOne ( void );
\end{lstlisting}
    \caption{Ekvivalent kontraktu bez klauzule \emph{assigns}}
    \label{fig:kontrakt-bez-ensures}
\end{figure}

V rámci tejto ukážky sme použili kľúčové slovo \emph{\textbackslash nothing}, ktoré si bližšie vysvetlíme neskôr v kapitole \ref{chapter:nothing}.

\subsection{allocates}

Jedná sa o kľúčové slovo využívané v rámci kontraktu funkcie. Pomocou tohto kľúčového slova možno špecifikovať množinu ukazateľov, ktoré môžu byť v rámci funkcie alokované (môžu, \textbf{nemusia}). Ostatné adresy nemôžu zmeniť svoj alokačný stav, ktorý môže byť napríklad statický, register, dynamický alebo nealokovaný. \cite{alokacne-stavy-acsl}
% TODO:
    % je to správne takto citovať, keď citujem len poslednú vetu?

Využitie tejto klauzule si ukážeme na funkcií, ktorej argumentom bude ukazateľ na celočíselné hodnoty. V rámci tejto funkcie bude tento ukazateľ alokovaný.

\begin{figure}[H]
    \centering
    \captionsetup{justification=centering}
\begin{lstlisting}[language=C]
//@ allocates * number;
void allocatePointer ( int * number );
\end{lstlisting}
    \caption{Ukážka klauzule \emph{allocates}}
    \label{fig:kontrakt-allocates}
\end{figure}

V prípade, že táto klauzula nie je použitá predpokladá sa, že v rámci funkcie nenastáva zmena alokačného stavu žiadnej pamäťovej bunky. To môžeme tiež ekvivalentne zapísať za použitia kľúčového slova \emph{\textbackslash nothing} (viď ukážka \ref{fig:kontrakt-bez-allocates}).

\begin{figure}[H]
    \centering
    \captionsetup{justification=centering}
\begin{lstlisting}[language=C]
//@ allocates \nothing;
void printOne ( void );
\end{lstlisting}
    \caption{Ekvivalent kontraktu bez klauzule \emph{allocates}}
    \label{fig:kontrakt-bez-allocates}
\end{figure}

\subsection{frees}

Kľúčové slovo \emph{frees} sa využíva v rámci kontraktu funkcie. Jedná sa o doplnok kľúčového slova \emph{allocates} -- je to v podstate jeho opak. V rámci tohto kľúčového slova možno špecifikovať množinu ukazateľov, ktorých pamäťové bunky sa behom funkcie uvoľnia.

Túto klauzulu si možno ukázať priamo na štandardnej funkcií jazyku C a to \emph{frees}, ktorá uvolní pamäť, na ktorú ukazuje ukazateľ v argumente funkcie.

\begin{figure}[H]
    \centering
    \captionsetup{justification=centering}
\begin{lstlisting}[language=C]
//@ frees * pointer;
void free ( void * pointer );
\end{lstlisting}
    \caption{Ukážka klauzule \emph{frees}}
    \label{fig:kontrakt-frees}
\end{figure}

Zároveň rovnako ako v prípade klauzule \emph{allocates}, ak funkcia žiadnu pamäť neuvoľňuje je možné túto klauzulu vynechať. Prípadne je možné priamo špecifikovať, že sa nič neuvoľňuje.

\begin{figure}[H]
    \centering
    \captionsetup{justification=centering}
\begin{lstlisting}[language=C]
//@ frees \nothing;
void printOne ( void );
\end{lstlisting}
    \caption{Ekvivalent kontraktu bez klauzule \emph{frees}}
    \label{fig:kontrakt-bez-frees}
\end{figure}

\subsection{predicate}

\emph{Predicate} (v preklade predikát) je z hladiska logiky logický výrazy, o ktorom možno rozhodnúť či je pravdivý, alebo nepravdivý. \cite{predikat-wiki} Jazyk \emph{ACSL} používa pre popis vstupných a výstupných podmienok logické výrazy. Tieto logické výrazy sa môžu opakovať -- môžu byť použité ako vstupné aj výstupné podmienky alebo vo viacerých funkciách. Kvôli takýmto prípadom je vhodné znižovať redundanciu, čo možno docieliť práve použitím kľúčového slova \emph{predicate}. 

Toto kľúčové slovo nám teda umožňuje špecifikovať pomenované logické výrazy obsahujúce rôzne premenné (argumenty), ktoré možno následne využívať v rámci kontraktov funkcií a tak sprehľadniť kód a znížiť jeho redundanciu. 

Túto klauzulu si ukážeme na funkcii, ktorá zmení hodnotu argumentu na absolutnú hodnotu jeho druhej odmocniny. Naviac keďže je v obore reálnych čísiel definovaná len pre nezáporné hodnoty budeme požiadavať, aby bola hodnota argumentu pred aj po vykonaní funkcie nezáporná. To docielime použitím kľúčových slov \emph{requires} a \emph{ensures}, no pre zníženie redundancie si zadefinueme predikát \emph{is\_nonnegative}, ktorý bude rozhodovať či je jeho argument nezáporné číslo (viď ukážka \ref{fig:predicate}).

\begin{figure}[H]
    \centering
    \captionsetup{justification=centering}
\begin{lstlisting}[language=C]
/*@ predicate is_nonnegative ( integer number ) = (
  @     number >= 0
  @ );
*/

/*@
  @ requires is_nonnegative ( * number );
  @
  @ ensures is_nonnegative ( * number );
*/
void squareRoot ( int * number );
\end{lstlisting}
    \caption{Ukážka klauzule \emph{predicate}}
    \label{fig:predicate}
\end{figure}

\subsection{logic}

Kľúčové slovo \emph{logic} umožňuje v rámci jazyku \emph{ACSL} špecifikovať jednoduché funkcie, ktoré možno využiť pre tvorbu predikátov a logických výrazov. Tieto funkcie nemôžu využivať premenné, ani cykly. Pre zložitejšie konštrukcie je možné použiť rekurziu.

Príkladom logickej funkcie môže byť napríklad získanie maxima v celočíselnom poli (viď ukážka \ref{fig:logic}).

\begin{figure}[H]
    \centering
    \captionsetup{justification=centering}
\begin{lstlisting}[language=C]
/*@ logic integer max_in_array ( integer * numbers, 
  @                              integer length ) = (
  @     ( length == 1 )
  @     ?
  @     numbers [ 0 ]
  @     :
  @     max ( numbers [ 0 ], 
  @           max_in_array ( numbers + 1, length - 1 ) 
  @         )
  @ );
*/
\end{lstlisting}
    \caption{Ukážka klauzule \emph{logic}}
    \label{fig:logic}
\end{figure}

\subsection{loop variant}

TODO

\subsection{loop invariant}

TODO

\subsection{loop assigns}

TODO

\subsection{behaviors}

TODO

\subsubsection{behavior}

TODO

\subsubsection{complete behaviors}

TODO

\subsubsection{disjoint behaviors}

TODO pre úplnosť

\subsection{assumes}

TODO

\subsection{ghost}

TODO

\subsection{\textbackslash result} \label{chapter:result}

TODO

\subsection{\textbackslash null}

TODO

\subsection{\textbackslash nothing} \label{chapter:nothing}

TODO

\subsection{\textbackslash valid}

TODO

\subsection{\textbackslash old} \label{chapter:old}

TODO

\subsection{\textbackslash at} 

TODO

\subsection{\textbackslash valid} 

TODO

\subsection{\textbackslash forall, \textbackslash exists}

TODO

\section{Pluginy} \label{chapter:pluginy}

TODO

\subsection{WP}

TODO

\subsection{EVA}

TODO

\subsection{RTE}

TODO
