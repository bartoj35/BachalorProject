% arara: pdflatex
% arara: pdflatex
% arara: pdflatex

% options:
% thesis=B bachelor's thesis
% thesis=M master's thesis
% czech thesis in Czech language
% slovak thesis in Slovak language
% english thesis in English language
% hidelinks remove colour boxes around hyperlinks
% 10pt 11pt 12pt

\documentclass[thesis=B,slovak]{FITthesis}[2019/03/21]

\usepackage[main=slovak,english]{babel}
\usepackage[utf8]{inputenc} % LaTeX source encoded as UTF-8
\usepackage{caption}
\usepackage{float}

\usepackage{tikz}
\usetikzlibrary{arrows}

\usepackage{color}
\definecolor{comments}{rgb}{0,0.70,0.35}
\definecolor{background}{rgb}{1,1,0.89}
\definecolor{numbers}{rgb}{0.5,0.5,0.5}
\definecolor{strings}{rgb}{0.58,0,0.82}

\usepackage{listings}
\lstdefinestyle{mystyle}{
    backgroundcolor=\color{background},   
    commentstyle=\color{comments},
    keywordstyle=\color{magenta},
    numberstyle=\tiny\color{numbers},
    stringstyle=\color{strings},
    basicstyle=\ttfamily\footnotesize,
    breakatwhitespace=false,         
    breaklines=true,                 
    captionpos=b,                    
    keepspaces=true,                 
    numbers=left,                    
    numbersep=5pt,                  
    showspaces=false,                
    showstringspaces=false,
    showtabs=false,                  
    tabsize=4
}
\lstset{style=mystyle}

% \usepackage{amsmath} %advanced maths
% \usepackage{amssymb} %additional math symbols

\usepackage{dirtree} %directory tree visualisation

% % list of acronyms
% \usepackage[acronym,nonumberlist,toc,numberedsection=autolabel]{glossaries}
% \iflanguage{czech}{\renewcommand*{\acronymname}{Seznam pou{\v z}it{\' y}ch zkratek}}{}
% \makeglossaries

% % % % % % % % % % % % % % % % % % % % % % % % % % % % % % 
% ODTIALTO DALEJ VSETKO ZMENTE
% % % % % % % % % % % % % % % % % % % % % % % % % % % % % % 

\department{Katedra informační bezpečnosti}
\title{Ověřená implementace struktury Union-Find}
\authorGN{Jakub} %(krstné) meno (mená) autora
\authorFN{Bartoň} % priezvisko autora
\authorWithDegrees{Jakub Bartoň} %meno autora včetne súčasných akademických titulov
\author{Jakub Bartoň} %meno autora bez súčasných akademických titulov
\supervisor{doc. RNDr. Dušan Knop, Ph.D.}
\acknowledgements{Doplňte, ak chcete niekomu za niečo poďakovať. V~opačnom prípade úplne odstráňte tento príkaz.}
\abstractCS{Union-Find je dátová štruktúra používaná v~úlohách, ktoré vyžadujú množinovú operáciu zjednotenia a identifikáciu, do akej množiny prvok patrí. Programátori často tieto štruktúry neimplementujú sami, ale vyhľadavajú ich implementácie online. V~takom prípade je problémom overenie, že implementácia funguje správne a~že vykonáva iba to čo má. Táto práca sa preto zameriava na analýzu dátovej štruktúry Union-Find, možnostmi jej implementácie, verifikáciou a~porovnaním výkonnosti jednotlivých implementácií.}

\abstractEN{Union-Find is a data structure that can be used for tasks that require set union and checking to which set an element belongs. Software developers do not implement those data structures on their own. They prefer to look for them online. The problem is, that it is difficult to verify the correctness of the implementation. This thesis focuses on analyzing the Union-Find data structure, ways of optimized implementation, verification, and performance comparison of selected implementations.}

\placeForDeclarationOfAuthenticity{V~Prahe}
\declarationOfAuthenticityOption{1} %voľba Prehlásenia (číslo 1-6)
\keywordsCS{dátová štruktura, verifikácia, Union-Find, Frama-C, symbolic engine, benchmark}
\keywordsEN{data structure, verification, union-find, Frama-C, symbolic engine, benchmark}
% \website{http://site.example/thesis} %volitelná URL práce, objeví se v tiráži

\begin{document}

% \newacronym{CVUT}{{\v C}VUT}{{\v C}esk{\' e} vysok{\' e} u{\v c}en{\' i} technick{\' e} v Praze}
% \newacronym{FIT}{FIT}{Fakulta informa{\v c}n{\' i}ch technologi{\' i}}

\begin{introduction}

V~dnešnej dobe veľká časť programátorov implementuje v~rámci svojej práce riešenia na konkrétne problémy, ale nie dátové štruktúry pre to potrebné. V~moment keď narazia na potrebu použitia dátovej štruktúry využívajú v~minulosti použité implementácie alebo si implementáciu nájdu online (napríklad na stránkach ako je github alebo stackoverflow). 

Sú však tieto implementacie správne? Môžeme sa naozaj spoľahnúť, že neobsahujú žiadne chyby, ktoré by mohli viesť k~nesprávnym výsledkom alebo v~horšom prípade eskalovať k~väčším problémom? Na tieto stránky predsa môže prispieť ktokoľvek. Kód, ktorý možno na týchto stránkach nájsť nie je nijako regulovaný, jediný spôsob pre zistenie či sa týka o~dobrý zdroj je popularita autora (počet ľudí sledujúcich užívateľa na githube alebo reputácia v~prípade stackoverflow). Aby sme však zaistili, že kód spĺňa naše požiadavky je potreba ho poriadne testovať. 

Testovanie je možné vykonať napriklad pomocou jednotkových testov (tzv. unit testov) alebo testom komponenty. Prípadne možno zvoliť nástroje typu symbolic engine \ref{symbolic_engine} -- ten nám umožní otestovať volania funkcií za pomoci symbolických premenných, ktoré menia svoju hodnotu, aby umožnili vykonať všetky možné priechody funkciou. Tento proces môže byť z~dôvodou častého vetvenia časovo a~výpočetne veľmi náročný a~tiež objaví skôr chyby poškodzujúce pamäť -- chyby vznikajúce nesprávnou prácou s~pamäťou (napríklad prístup mimo pole, či už na zásobníku, alebo na halde alebo dereferencia pamäti, ku~ktorej by sme nemali mať prístup). 

Existuje však nejaké vhodnejšie riešenie? Našťastie odpoveď je áno. Tomuto riešeniu sa hovorí verifikácia kódu. Jedná sa o~proces, ktorý nám umožňuje formálne dokázať, že náš kód funguje správne a~vykonáva práve to, čo od neho očakávame a~to za pomoci software a presne špecifikovaných informácií ako by mali vyzerať parametry funkcie, ako by mal vyzerať výsledok funkcie atď. Verifikáciu možno vykonať pomocou rôznych nástrojov a~frameworkov, príkladom pre jazyky C je framework Frama-C \ref{frama-c_uvod}.

\section{Cieľ práce}

V~rámci tejto práce sa oboznámime s~dátovou štruktúrou Union-Find \ref{union-find_uvod}, zistíme aké sú možnosti implementácie tejto štruktúry pre dosiahnutie zrýchlenia. V~programovacom jazyku C vytvoríme implementáciu všetkých týchto spôsobov. Zoznámime sa s~frameworkom Frama-C \ref{frama-c_uvod}, ktorý slúži pre verifikáciu programov v~programovacom jazyku C, následne ho využijeme pre verifikáciu nami vytvorených implementácií dátovej štruktúry. Overíme správnosť našej implementácie testovaním \ref{testovanie} a~tiež sa pozrieme na výkonnostné rozdiely naprieč implementáciami \ref{benchmark}.

\end{introduction}

\chapter{Union-Find} 

V~tejto sekcií popíšeme aké operácie Union-Find podporuje, aké sú možnosti využitia tejto dátovej štruktúry, vysvetlíme si ako sme v~rámci tejto práce reprezentovali štruktúru, ako sa operácie tejto štruktúry implementujú a~aké sú možnosti ich optimalizácie.

Pre potreby vysvetlenia akou dátovou štruktúrou Union-Find je a pre vysvetlenie reprezentácie tejto štuktúry je potrebné následujúce názvoslovie:
\begin{itemize}
    \item nezávislé (disjunktné) množiny -- dve množiny nazveme nazávislými práve vtedy, keď ich prienik tvorí prázdnu množinu,
    \item \textbf{grafové názvoslovie...}
\end{itemize}

% TODO:
    % doplň z teórie grafov a pridaj citácie
    % graf, strom, zakorenený strom, cesta, koreň, súvislosť, slučka, stok

\section{Čo to je Union-Find?} \label{union-find_uvod}

Union-Find je dátová štruktúra zložená z~kolekcie nezávislých množín, kde každá množina je identifikovaná jednoznačným identifikátorom. Často je možné sa s~touto dátovou štruktúrou stretnúť aj pod názvom \emph{Merge-Find} alebo \emph{Disjoint Set}.

\begin{figure}[H]
    \centering
    \captionsetup{justification=centering}
    \begin{tikzpicture}
        \tikzset{set/.style = {draw,circle,minimum size=7.5em}}
        \tikzset{element/.style = {draw,circle,fill=gray,text=white}}
        
        \node[set] (0) at (0,0) {0};
        \node[set] (4) at (5,0) {4};

        \node[element] (0) at (-0.8,0) {0};
        \node[element] (1) at (0,-0.8) {1};
        \node[element] (2) at (0.8,0) {2};
        \node[element] (3) at (0,0.8) {3};

        \node[element] (4) at (5.6,0) {4};
        \node[element] (5) at (4.7,-0.7) {5};
        \node[element] (6) at (4.5,0.6) {6};
    \end{tikzpicture}
    \caption{Vennove diagramy reprezentujúce 2 disjunktné množiny s~identifikátormi 1 a 2}
    \label{fig:venn}
\end{figure}

Union-Find podporuje následujúce 3 operácie:

\newpage

\begin{itemize}
    \item pridanie prvku,
    \begin{itemize}
        \item tento novo vložený prvok reprezentuje novú množinu obsahujúcu iba tento prvok,
        \item ako napovedá jeden z~názvov tejto dátovej štruktúry (\emph{Disjoint Set}, v~preklade \emph{disjunktna množina}), štruktúra \textbf{neumožňuje} vkládanie prvkov, ktoré sa už v~niektorej z~množín nachádzajú,
    \end{itemize}
    \item zjednotenie množín
    \begin{itemize}
        \item umožňuje tvorbu množiny obsahujúcej všetky prvky pôvodných 2 množín (a~zánik pôvodných 2 množín),
    \end{itemize}
    \item identifikácia množiny, do ktorej prvok patrí,
    \begin{itemize}
        \item táto operácia umožňuje získať identifikátor množiny, do ktorej prvok patrí.
    \end{itemize}
\end{itemize}

\section{Aké je využitie?}

Táto dátova štruktúra podporuje často požadované operácie pridanie nového prvku, zjednotenia množín a~identifikácie, do ktorej množiny prvok patrí. 

Príkladom takýchto problémov a~algoritmov sú:
    
\begin{itemize}
    \item Kruskalov algoritmus pre hľadanie minimálnej kostry \cite{kruskal},
    \item hľadanie komponent súvislosti,
    \item detekcia cyklu v~grafe.
\end{itemize}

\section{Aké sú možnosti implementácie?}

Nakoľko nemá táto dátová štruktúra presne špecifikovaný spôsob ukladania dát je to na nás. My sa budeme konkrétne držať reprezentácie štruktúry ako kolekcie orientovaných zakoreňených stromov, kde každý strom predstavuje jednu množínu. \cite{union-find-kriky}

Tieto stromy obsahujú orientované hrany tvoriace cestu do koreňa stromu, ktorého hodnota jednoznačne identifikuje množinu (toto možno využiť na základe znalosti, že Union-Set neumožňuje vkladanie duplicitných prvkov). Táto skutočnosť nám umožňuje jednoduchú identifikáciu množiny a~to pomocou nájdenia koreňa stromu.

\begin{figure}[H]
    \centering
    \captionsetup{justification=centering}
    \begin{tikzpicture}
        \tikzset{vertex/.style = {draw, circle, minimum size = 3em, node distance = 4em and 4em}}
        \tikzset{edge/.style = {->,> = latex'}}
        
        \node[vertex] (0) at  (1,0) {0};
        \node[vertex] (3) [below left of=0] {3};
        \node[vertex] (2) [below right of=0] {2};
        \node[vertex] (1) [below left of=3] {1};
        
        \node[vertex] (4) at (7,0) {4};
        \node[vertex] (5) [below left of=4] {5};
        \node[vertex] (6) [below right of=4] {6};
        
        \draw[edge] (3) to (0);
        \draw[edge] (2) to (0);
        \draw[edge] (1) to (3);
        
        \draw[edge] (5) to (4);
        \draw[edge] (6) to (4);
    \end{tikzpicture}    
    
    \caption{Kolekcia zakoreňených stromov reprezentujúcich množiny vennovho diagramu \ref{fig:venn}}
    \label{fig:stromy}
\end{figure}

Orientované hrany zároveň predstavujú vzťah zjednotenia dvoch množín a~to ako tieto hrany budú konštruované záleží na implementácií operácie zjednotenia a~prípadne operácie nájdenia identifikátoru množiny, ktorá v~niektorých optimalizovaných verziách manipuluje s~hranami stromu množiny. Možností implementácií týchto funkcionalít na stromovej reprezentácií si vysvetlíme v~následujúcich podkapitolách.

\subsection{Základná implementacia}

Základná implementácia nevyužíva žiadne urýchlenie jednotlivých operácií, jedná sa o~veľmi jednoduché riešenie, kde získanie identifikátoru množiny, do ktorej prvok patrí je vykonané pomocou priechodu cesty od prvku ku~koreňu, ktorý reprezentuje identifikátor množiny. 

Nájdenie identifikátoru je vykonané iba na základe nájdenia koreňa stromu (to znamená priechodu cesty, až do doby kým nájdeme stok).

Zjednotenie dvoch prvkov z~rôznych množín prebieha následovne:

\begin{enumerate}
    \item Nájdeme identifikátory množín oboch prvkov.
    \item Vykonáme kontrolu či sa nepokúšame zjednotiť identické množiny (pomocou identifikátorov množín).
    \item Vytvoríme hranu medzi koreňom stromu reprezentujúceho množinu, do ktorej patrí druhý prvok a prvkom prvým.
\end{enumerate}

\begin{figure}[H]
    \centering
    \captionsetup{justification=centering}
    \begin{tikzpicture}
        \tikzset{vertex/.style = {draw, circle, minimum size = 3em, node distance = 4em and 4em}}
        \tikzset{edge/.style = {->,> = latex'}}
        
        \node[vertex] (0) at  (1,0) {0};
        \node[vertex] (3) [below left of=0] {3};
        \node[vertex] (2) [below right of=0] {2};
        \node[vertex] (1) [below left of=3] {1};
        
        \node[vertex] (4) [below right of=2] {4};
        \node[vertex] (5) [below left of=4] {5};
        \node[vertex] (6) [below right of=4] {6};
        
        \draw[edge] (3) to (0);
        \draw[edge] (2) to (0);
        \draw[edge] (1) to (3);
        
        \draw[edge] (5) to (4);
        \draw[edge] (6) to (4);
        \draw[edge] (4) to (2);
    \end{tikzpicture}
         
    \caption{Výsledok volania \emph{union(2,6)} na množiny reprezentovanej v~\ref{fig:stromy} pri využití základnej implementácií}
    \label{fig:priklad_zakladna_implementacia}
\end{figure}

Ako si možno všimnúť na obrázku vyššie táto implementácia žiadnym spôsobom nerieši hĺbku stromu a~teda operácia nájdenia identifikátoru množiny môže trvať neprimerane dlho (až $O(n)$), preto sa ďalej pozrieme ako možno operácie implementovať, aby sme dosiahli lepších výsledkov z~hladiska rýchlosti operácií.

\subsection{Zjednotenie podľa rádu}

Táto implementácia čiastočne optimalizuje hĺbku stromu reprezentujúceho zjednotené množiny pomocou pravidla, ktoré určuje akým spôsobom sa stromy množín prepoja. Nájdenie identifikátoru množiny prebieha pozostáva bez zmien.

Toto pravidlo konkrétne hovorí o~tom, že pri zjednotení množín obsahujúcich prvky, ktoré chceme zjednotiť prepojíme vytvorením hrany z~koreňa stromu menšieho rádu do koreňa stromu vyššieho rádu (na orientácií tejto hrany nezáleží, ak sa jedná o~stromy rovnakého rádu). 

Tým spôsobíme to, že nový strom je maximálne rádu o~1 väčšieho než maximum z~pôvodných dvoch stromov (a~teda aj maximálne o~1 hlbší ako pôvodný strom). Zatiaľ čo v~základnej implementácií by vznikal strom, ktorého hĺbka je súčtom hĺbok oboch stromov. Vďaka tejto skutočnosti bude vo~veľkej časti prípadov hĺbka stromu menšia ako pri použití základnej implementácie a~teda aj pri operácií hľadania identifikátoru množiny sa vykoná menej operácií.

Zjednotenie dvoch prvkov z~rôznych množín prebieha následovne:

\begin{enumerate}
    \item Nájdeme identifikátory množín oboch prvkov.
    \item Vykonáme kontrolu či sa nepokúšame zjednotiť identické množiny (pomocou identifikátorov množín).
    \item Vytvoríme orientovanú hranu z~koreňa stromu nižšieho rádu do koreňa stromu vyššieho rádu (ak rády boli identické na orientácií hrany nezáleží).
    \item Ak boli stromy identického rádu upravíme rád novovzniknutého stromu.
\end{enumerate}

\begin{figure}[H]
    \centering
    \captionsetup{justification=centering}
    \begin{tikzpicture}
        \tikzset{vertex/.style = {draw, circle, minimum size = 3em, node distance = 5em and 5em}}
        \tikzset{edge/.style = {->,> = latex'}}
        
        \node[vertex] (0) at  (1,0) {0};
        \node[vertex] (3) [below left of=0] {3};
        \node[vertex] (2) [below of=0] {2};
        \node[vertex] (1) [below left of=3] {1};
        
        \node[vertex] (4) [below right of=0] {4};
        \node[vertex] (5) [below of=4] {5};
        \node[vertex] (6) [below right of=4] {6};
        
        \draw[edge] (3) to (0);
        \draw[edge] (2) to (0);
        \draw[edge] (1) to (3);
        
        \draw[edge] (5) to (4);
        \draw[edge] (6) to (4);
        \draw[edge] (4) to (0);
    \end{tikzpicture}
         
    \caption{Výsledok volania \emph{union(2,6)} na množiny reprezentovanej v~\ref{fig:stromy} pri implementácií zjednotenia podľa rádu}
    \label{fig:priklad_zjednotenie_podľa_rádu}
\end{figure}

\subsection{Zjednotenie podľa veľkosti}

Tento spôsob rovnako ako zjednotenie podľa rádu upravuje zjednotenie dvoch množín obsahujúcich požadované prvky a~nájdenie identifikátoru množiny zostáva bez zmien.

Modifikacia zjednotenia spočíva vo~vzniku pravidla, ktoré určuje ako vznikne hrana medzi dvomi stromami množín po zjednotení. Jediný rozdieľ od zjednotenia na základe rádu je v~tom, že tentokrát je pre nás dôležitá informácia o~veľkosti stromu (počet prvkov množiny). Táto modifikácia spôsobuje to, že nový strom bude mať vo~väčšine prípadou menšiu hĺbku ako po zjednotení pomocou základnej implementácie a~teda aj pri operácií hľadania identifikátoru množiny sa vykoná menej operácií.

\begin{figure}[H]
    \centering
    \captionsetup{justification=centering}
    \begin{tikzpicture}
        \tikzset{vertex/.style = {draw, circle, minimum size = 3em, node distance = 5em and 5em}}
        \tikzset{edge/.style = {->,> = latex'}}
        
        \node[vertex] (0) at  (1,0) {0};
        \node[vertex] (3) [below left of=0] {3};
        \node[vertex] (2) [below of=0] {2};
        \node[vertex] (1) [below left of=3] {1};
        
        \node[vertex] (4) [below right of=0] {4};
        \node[vertex] (5) [below of=4] {5};
        \node[vertex] (6) [below right of=4] {6};
        
        \draw[edge] (3) to (0);
        \draw[edge] (2) to (0);
        \draw[edge] (1) to (3);
        
        \draw[edge] (5) to (4);
        \draw[edge] (6) to (4);
        \draw[edge] (4) to (0);
    \end{tikzpicture}
         
    \caption{Výsledok volania \emph{union(2,6)} na množiny reprezentovanej v~\ref{fig:stromy} pri implementácií zjednotenia podľa rádu}
    \label{fig:priklad_zjednotenie_podľa_veľkosti}
\end{figure}

Táto implementácia však umožňuje tvorbu stromov väčšej hĺbky o~malom počte prvkov, tým pádom aj zhoršovať rýchlosť nájdenia identifikátoru množiny v~porovnaní so~zjednotením podľa hĺbky a~to napríklad v~prípade existencie stromu obsahujúceho tri vrcholy, ale hĺbkou 3 a~stromu obsahujúceho štyri vrcholoch, ale hĺbkou 2.

\begin{figure}[H]
    \centering
    \captionsetup{justification=centering}
    \begin{tikzpicture}[scale=0.45]
        \tikzset{vertex/.style = {draw, circle, minimum size = 3em, node distance = 5em and 5em}}
        \tikzset{edge/.style = {->,> = latex'}}
        
        \node[vertex] (0) {0};
        \node[vertex] (1) [below left of=0] {1};
        \node[vertex] (2) [below of=0] {2};
        \node[vertex] (3) [below right of=0] {3};
        
        \node[vertex] (5) [right of=3] {5};
        \node[vertex] (4) [above right of=5] {4};
        \node[vertex] (6) [below right of=5] {6};
        
        \draw[edge] (3) to (0);
        \draw[edge] (2) to (0);
        \draw[edge] (1) to (0);
        
        \draw[edge] (5) to (4);
        \draw[edge] (6) to (5);
    \end{tikzpicture}
    
    \caption{Príklad stromov reprezentujúcich množiny, ktoré pri zjednotení podľa hĺbky budú mať menšiu hĺbku ako pri zjednotení podľa veľkosti}
    \label{fig:zly_strom_size}
\end{figure}

\begin{figure}[H]
    \centering
    \captionsetup{justification=centering}
    \begin{tikzpicture}[scale=0.45]
        \tikzset{vertex/.style = {draw, circle, minimum size = 3em, node distance = 5em and 5em}}
        \tikzset{edge/.style = {->,> = latex'}}
        
        \node[vertex] (0) {0};
        \node[vertex] (1) [below left of=0] {1};
        \node[vertex] (2) [below of=0] {2};
        \node[vertex] (3) [below right of=0] {3};
        
        \node[vertex] (4) [right of=0] {4};
        \node[vertex] (5) [below right of=4] {5};
        \node[vertex] (6) [below of=5] {6};
        
        \draw[edge] (3) to (0);
        \draw[edge] (2) to (0);
        \draw[edge] (1) to (0);
        
        \draw[edge] (4) to (0);
        \draw[edge] (5) to (4);
        \draw[edge] (6) to (5);
    \end{tikzpicture}
         
    \caption{Výsledok volania \emph{union(0,4)} na množiny reprezentovanej v~\ref{fig:zly_strom_size} pri implementácií zjednotenia podľa veľkosti}
    \label{fig:zly_priklad_zjednotenie_podľa_veľkosti}
\end{figure}

\subsection{Kompresia cesty}

Táto implementácia na rozdieľ od predošlých optimalizácií mení implementáciu operácie nájdenia identifikátoru množiny, nie zjednotenia, tá prebieha rovnako ako v~prípade základnej implementácie.

Modifikácia nájdenia identifikátoru množiny je taká, že po nájdení identifikátoru sú všetky hrany vrcholov nachádzajúcich sa na ceste z~vrcholu obsahujúci prvok, ktorého množiny sme chceli zistiť identifikátor, nahradené hranou do koreňa množiny.

Tým redukujeme dĺžky ciest v~rámci stromu, prípadne to môže viesť až k~redukcií hĺbky stromu (za predpokladu, že sa pokúšame nájsť identifikátor na základe prvku, ktorý sa nachádza na najdlhšej ceste v~strome, viz. obrázok \ref{fig:priklad_kompresia_cesty}).

Nájdenie identifikátoru prebieha teda následovne:
\begin{enumerate}
    \item Nájdeme koreň stromu, obsahujúceho prvok.
    \item Prejdeme cestu od vrcholu obsahujúceho prvok do koreňa znovu a~každému vrcholu nahradíme pôvodnú hranu hranou do koreňa.
\end{enumerate}

Pre názornosť operácie \emph{find} v~následujúcich implementáciach použijeme množinu reprezentovanú následovne:

\begin{figure}[H]
    \centering
    \captionsetup{justification=centering}
    \begin{tikzpicture}
        \tikzset{vertex/.style = {draw, circle, minimum size = 3em, node distance = 4em and 4em}}
        \tikzset{edge/.style = {->,> = latex'}}
        
        \node[vertex] (0) at (1,0) {0};
        \node[vertex] (1) [right of=0] {1};
        \node[vertex] (2) [right of=1] {2};
        \node[vertex] (3) [right of=2] {3};
        \node[vertex] (4) [right of=3] {4};
        \node[vertex] (5) [right of=4] {5};
        \node[vertex] (6) [right of=5] {6};
        
        \draw[edge] (1) to (0);
        \draw[edge] (2) to (1);
        \draw[edge] (3) to (2);
        \draw[edge] (4) to (3);
        \draw[edge] (5) to (4);
        \draw[edge] (6) to (5);
    \end{tikzpicture}
         
    \caption{Množina reprezentovaná ako cesta}
    \label{fig:cesta}
\end{figure}

\begin{figure}[H]
    \centering
    \captionsetup{justification=centering}
    \begin{tikzpicture}
        \tikzset{vertex/.style = {draw, circle, minimum size = 3em, node distance = 5em and 5em}}
        \tikzset{edge/.style = {->,> = latex'}}
        
        \node[vertex] (0) at (1,0) {0};
        \node[vertex] (1) [above left of=0] {1};
        \node[vertex] (2) [left of=0] {2};
        \node[vertex] (3) [below left of=0] {3};
        \node[vertex] (4) [below of=0] {4};
        \node[vertex] (5) [right of=0] {5};
        \node[vertex] (6) [right of=5] {6};
        
        \draw[edge] (1) to (0);
        \draw[edge] (2) to (0);
        \draw[edge] (3) to (0);
        \draw[edge] (4) to (0);
        \draw[edge] (5) to (0);
        \draw[edge] (6) to (5);
    \end{tikzpicture}
         
    \caption{Výsledok volania \emph{find(5)} na množine reprezentovanej v~\ref{fig:cesta} pri implementácií kompresie cesty}
    \label{fig:priklad_kompresia_cesty}
\end{figure}

\subsection{Delenie cesty}

Táto implementácia rovnako ako kompresia cesty modifikuje operáciu nájdenia identifikátoru množiny.

Delenie cesty na rozdieľ od kompresie cesty dĺžku cesty z~vrcholu do koreňa zmenšuje, ale neminimalizuje (to môže nastať iba v~prípade viacnásobného využitia). Delenie cesty po nájdení identifikátoru znovu prejde celú cestu a~každému vrcholu zmení cieľ jeho hrany do vrcholu, do ktorého vedie hrana jej pôvodného cieľa (viz. obrázok \ref{fig:priklad_delenie_cesty}).

\begin{figure}[H]
    \centering
    \captionsetup{justification=centering}
    \begin{tikzpicture}
        \tikzset{vertex/.style = {draw, circle, minimum size = 3em, node distance = 4em and 4em}}
        \tikzset{edge/.style = {->,> = latex'}}
        
        \node[vertex] (0) at (1,0) {0};
        \node[vertex] (1) [below left of=0] {1};
        \node[vertex] (2) [below right of=0] {2};
        \node[vertex] (3) [below left of=1] {3};
        \node[vertex] (4) [below right of=2] {4};
        \node[vertex] (5) [below left of=3] {5};
        \node[vertex] (6) [below right of=4] {6};
        
        \draw[edge] (1) to (0);
        \draw[edge] (2) to (0);
        \draw[edge] (3) to (1);
        \draw[edge] (4) to (2);
        \draw[edge] (5) to (3);
        \draw[edge] (6) to (4);
    \end{tikzpicture}
         
    \caption{Výsledok volania \emph{find(6)} na množine reprezentovanej v~\ref{fig:cesta} pri implementácií delenia cesty}
    \label{fig:priklad_delenie_cesty}
\end{figure}

\subsection{Pólenie cesty}

Táto implementácia je veľmi podobná optimalizácií pomocou delenia cesty. Rozdiel je ten, že pri pólení ciest vykonávame zmenu hrany iba na každom druhom vrchole na ceste z~prvku, pre ktorého množinu sme vykonávali nájdenie identifikátoru a~koreňom jeho množiny (viz. obrázok \ref{fig:priklad_pólenia_cesty}). No v~prípade delenia cesty vykonávame zmenu hrany vrcholu v~každom vrchole na ceste do koreňa.

\begin{figure}[H]
    \centering
    \captionsetup{justification=centering}
    \begin{tikzpicture}
        \tikzset{vertex/.style = {draw, circle, minimum size = 3em, node distance = 4em and 4em}}
        \tikzset{edge/.style = {->,> = latex'}}
        
        \node[vertex] (0) at (1,0) {0};
        \node[vertex] (1) [below left of=0] {1};
        \node[vertex] (2) [below right of=0] {2};
        \node[vertex] (3) [below left of=2] {3};
        \node[vertex] (4) [below right of=2] {4};
        \node[vertex] (5) [below left of=4] {5};
        \node[vertex] (6) [below right of=4] {6};
        
        \draw[edge] (1) to (0);
        \draw[edge] (2) to (0);
        \draw[edge] (3) to (2);
        \draw[edge] (4) to (2);
        \draw[edge] (5) to (4);
        \draw[edge] (6) to (4);
    \end{tikzpicture}
         
    \caption{Výsledok volania \emph{find(6)} na množine reprezentovanej v~\ref{fig:cesta} pri implementácií pólenia cesty}
    \label{fig:priklad_pólenia_cesty}
\end{figure}


\chapter{Frama-C} 

V~tejto kapitole sa budeme venovať frameworku slúžiacemu pre verifikáciu zdrojových kódov v~programovacom jazyku C. Tým frameworkom je \emph{Frama-C}.

Vysvetlíme si, ako tento framework funguje, popíšeme si, čo je jeho základným stavebným kameňom, rozoberieme si význam a~využitie vybraných kľúčových slov jazyku \emph{ACSL}\footnote{ANSI/ISO C Specification Language} (jazyk používaný v~rámci frameworku) a~na záver sa pozrieme na pluginy, ktoré sme pri práci s frameworkom použili.

\section{Čo to je Frama-C?} \label{frama-c_uvod}

\emph{Frama-C} je open-source\footnote{zdrojové kódy sú verejné prístupné a~ktokoľvek ich môže modifikovať pri dodržaní licenčných práv} framework slúžiaci pre analýzu zdrojových kódov. Jeho použitím sa užívateľ snaží odhaliť neočakávaného chovania a~preukázať správnu funkčnosť zdrojového kódu alebo implementácie. Framework užívateľovi umožňuje definovať požiadavky na chovanie funkcionality a~pomocou nich ukázať, že jeho implementácia tieto požiadavky splňuje. Pri vhodnej tvorbe požiadavkov preto možno zaručiť, že funkcionalita neobsahuje chyby.\cite{frama-c-uvod}

Framework tiež podporuje pluginy\footnote{rozšírenia pridávajúce rôzne funkcionality}, ktoré môže byť často veľmi prospešné pri overovaní funkcionality programu. My sme sa rozhodli v rámci našej práce použiť niekoľko z~nich, konkrétne \emph{WP}, \emph{EVA} a \emph{RTE}. Tieto pluginy si bližšie popíšeme v~kapitole \ref{chapter:pluginy}.

\section{Kontrakt}

Framework pre špecifikáciu požiadavkov na funkcionality používa takzvaný kontrakt -- jedná sa o~základnú stavebnú jednotku tohto frameworku.

Kontrakt pozostává z~požiadavkov, ktoré by mali platiť pre jednotlivé funkcie, cykly či vetvenia a~nachádzajú sa v~anotačných komentároch. Tieto požiadavky sú popísané pomocou jazyku \emph{ACSL}. Ten obsahuje vstavané funkcie a~predikáty, datové typy premenných, takzvaný ghost code a~ghost premenné. \cite{obsah-acsl} Kontrakt sa bežne skladá zo~vstupných a~výstupných požiadavkov, pre ktoré sa používajú kľúčové slová \emph{requires} a~\emph{ensures}. Tieto kľúčové slová a~niekoľko ďaľších komponent tohto jazyku si vysvetlíme v~nasledujúcej časti tejto kapitoly.

\begin{figure}[H]
    \centering
    \captionsetup{justification=centering}
\begin{lstlisting}[language=C]
//@ requires <some_condition>;
void print ( int number );
\end{lstlisting}
    \caption{Ukážka jednoriadkového kontraktu}
    \label{fig:jednoduchy-kontrakt-jednoriadkovy}
\end{figure}

\begin{figure}[H]
    \centering
    \captionsetup{justification=centering}
\begin{lstlisting}[language=C]
/*@
  @ requires <some_condition>;
  @
  @ ensures <some_condition>;
*/
void append ( char ** destination, char * source );
\end{lstlisting}
    \caption{Ukážka viacriadkového kontraktu}
    \label{fig:jednoduchy-kontrakt-viacriadkovy}
\end{figure}

\section{Kľúčové slova jazyku ACSL}

Anotačný jazyk \emph{ACSL} slúži pre špecifikáciu požiadavkov na funkcie, cykly a vetvenia. Kľúčové slová môžu slúžiť pre označenie vstupný alebo výstupných podmienok, logických výrazov (v rámci nich možno použiť indikátory reprezentujúce hodnoty pravdy a~nepravdy vo~forme kľúčových slov \emph{\textbackslash true} a \emph{\textbackslash false}) a~funkcií, označenie kódu prístupného len v~rámci kontraktov alebo aj identifikátory stavu premenných (napríklad stav parametru pred a~po vykonaní funkcie).

Kľúčové slova jazyku \emph{ACSL} delíme na termy, predikáty a~zvyšné kľúčové slová. Termy a~predikáty sú špecifické tým, že na rozdiel od zvyšku sa ich názov začína spätným lomítkom (napríklad už zmienené \emph{\textbackslash true} a \emph{\textbackslash false}). \cite{keywords-acsl}

\subsection{requires} \label{chapter:requires}

Klauzula \emph{requires} sa využíva v~rámci kontraktu funkcie a~špecifikuje vstupné podmienky funkcie (napríklad aké musia byť hodnoty parametrov a~aká musí byť štruktúra ich dát). Pre pokračovanie v analýze požiadavkov na funkciu je potrebné, aby klauzula \emph{requires} bola splnená (ak vstupné dáta nesplňujú požiadavky, ktoré sme po nich požadovali nemôžme garantovať ako bude výsledok vyzerať). Zároveň v prípade väčšieho množstva požiadavkov je možné ich reťazenie pomocou operátoru logického súčinu (\&\&) alebo viackrát využiť túto klauzulu.

Použitie klauzule \emph{requires} si ukážeme na funkcií \emph{logarithm}, ktorá vypočíta hodnotu dekadického logaritmu pre hodnotu nachádzajúcu v~argumente. Nakoľko je logaritmus definovaný iba pre nezáporné hodnoty, požadujeme nezáporný argument aj v~rámci našej funkcie (viď kontrakt v ukážke \ref{fig:kontrakt-requires}).

\begin{figure}[H]
    \centering
    \captionsetup{justification=centering}
\begin{lstlisting}[language=C]
//@ requires number >= 0;
int logarithm ( int number );
\end{lstlisting}
    \caption{Príklad klauzule \emph{requires}}
    \label{fig:kontrakt-requires}
\end{figure}

Ak kontrakt neobsahuje túto klauzulu predpokladá sa, že vstupné požiadavky nie su žiadne.\cite{default-requires-acsl}. To si ukážeme na funkcií, ktorá na štandardný výstup vypíše číslo jeden (viď ukážka \ref{fig:kontrakt-bez-requires}).

\begin{figure}[H]
    \centering
    \captionsetup{justification=centering}
\begin{lstlisting}[language=C]
//@ requires \true;
void printOne ( void );
\end{lstlisting}
    \caption{Ekvivalent kontraktu bez klauzule \emph{requires}}
    \label{fig:kontrakt-bez-requires}
\end{figure}

\subsection{ensures} \label{chapter:ensures}

Klauzula \emph{ensures} dopĺňa v rámci kontraktu funkcie klauzulu \emph{requires}. Klauzula \emph{ensures} špecifikuje výstupné podmienky funkcie (napríklad aké podmienky musí splňovať návratová hodnota a~čo musí platiť pre vstupné parametry, \textbf{ale až na konci funkcie} -- napríklad v prípade zmeny hodnôt v~poli, ktoré bolo predané funkcií ako argument). Pre špecifikáciu viacerých výstupných podmienok možno využiť operátor logického súčinu (\&\&) alebo viackrát použiť túto klauzulu.

Použitie klauzule \emph{ensures} si ukážeme na funkcií \emph{square}, ktorá zmení hodnotu argumentu na hodnotu jej druhú mocniny. Keďže druhá mocnina ľubovolného čísla je číslo nezáporné, budeme požadovať nezápornú hodnotu argumentu na konci funkcie -- to dosiahneme pomocou klauzule \emph{ensures} (viď kontrakt v ukážke \ref{fig:kontrakt-ensures}).

\begin{figure}[H]
    \centering
    \captionsetup{justification=centering}
\begin{lstlisting}[language=C]
//@ ensures ( * number ) >= 0;
void square ( int * number );
\end{lstlisting}
    \caption{Ukážka klauzule \emph{requires}}
    \label{fig:kontrakt-ensures}
\end{figure}

Podobne ako v~prípade klauzule \emph{requires} sa pri vynechaní klauzule \emph{ensures} predpokladá, že na funkciu nie su kladené žiadne výstupné podmienky. Teda je táto možnosť ekvivalentná následujúcemu zápisu:

\begin{figure}[H]
    \centering
    \captionsetup{justification=centering}
\begin{lstlisting}[language=C]
//@ ensures \true;
void printOne ( void );
\end{lstlisting}
    \caption{Ekvivalent kontraktu bez klauzule \emph{ensures}}
    \label{fig:kontrakt-bez-ensures}
\end{figure}

Táto klauzula zároveň obsahuje jednu limitáciu a~to konkrétne v~prípade použitia kľúčového slova \textbf{goto}. Pri použití tohto kľúčového slova nie je klauzula \emph{ensures} overovaná. \cite{ensures-goto-acsl}

\subsection{assigns}

Toto kľúčové slovo sa využíva v rámci kontraktu funkcie a~špecifikuje, ktoré pamäťové bunky (premenné) sa môžu v~rámci funkcie meniť. V~prípade viacerých pamäťových buniek je možné využiť viacero klauzulí alebo ich vypísať ako zoznam viacerých premenných (v~takom prípade musia byť pamäťové bunky oddelené čiarkou). Pamäťové bunky, ktoré sa v~rámci tejto klauzule špecifikujú musia byť argumentom funkcie alebo globálne premenné \footnote{nemôže sa jednať o~lokálne premenné v~rámci funkcie nakoľko tieto premenné na začiatku funkcie ešte neexistujú}.

Použitie tejto klauzule si ukážeme na funkcií \emph{swap}, ktorá bude vykonávať výmenu obsahu 2 premenných typu int. Nakoľko funkcia zmení obsah pamäťových buniek vstupujúcich do funkcie je vhodné tieto zmeny v~rámci funkcie označiť a~to práve pomocou klauzule \emph{assings}.

\begin{figure}[H]
    \centering
    \captionsetup{justification=centering}
\begin{lstlisting}[language=C]
//@ assigns * number1, * number2;
void swap ( int * number1, int * number2 );
\end{lstlisting}
    \caption{Ukážka klauzule \emph{assings}}
    \label{fig:kontrakt-assigns}
\end{figure}

Nakoľko niektoré funkcie nemusia do takýchto pamäťových buniek zapisovať je možno túto klauzulu vynechať. Prípadne možno pre úplnosť špecifikovať, že sa nezapisuje do žiadnej takejto bunky následovne:

\begin{figure}[H]
    \centering
    \captionsetup{justification=centering}
\begin{lstlisting}[language=C]
//@ assigns \nothing;
void printOne ( void );
\end{lstlisting}
    \caption{Ekvivalent kontraktu bez klauzule \emph{assigns}}
    \label{fig:kontrakt-bez-ensures}
\end{figure}

V rámci tejto ukážky sme použili kľúčové slovo \emph{\textbackslash nothing}, ktoré si bližšie vysvetlíme neskôr v kapitole \ref{chapter:nothing}.

\subsection{allocates}

Jedná sa o kľúčové slovo využívané v rámci kontraktu funkcie. Pomocou tohto kľúčového slova možno špecifikovať množinu ukazateľov, ktoré môžu byť v rámci funkcie alokované (môžu, \textbf{nemusia}). Ostatné adresy nemôžu zmeniť svoj alokačný stav, ktorý môže byť napríklad statický, register, dynamický alebo nealokovaný. \cite{alokacne-stavy-acsl}
% TODO:
    % je to správne takto citovať, keď citujem len poslednú vetu?

Využitie tejto klauzule si ukážeme na funkcií, ktorej argumentom bude ukazateľ na celočíselné hodnoty. V rámci tejto funkcie bude tento ukazateľ alokovaný.

\begin{figure}[H]
    \centering
    \captionsetup{justification=centering}
\begin{lstlisting}[language=C]
//@ allocates * number;
void allocatePointer ( int * number );
\end{lstlisting}
    \caption{Ukážka klauzule \emph{allocates}}
    \label{fig:kontrakt-allocates}
\end{figure}

V prípade, že táto klauzula nie je použitá predpokladá sa, že v rámci funkcie nenastáva zmena alokačného stavu žiadnej pamäťovej bunky. To môžeme tiež ekvivalentne zapísať za použitia kľúčového slova \emph{\textbackslash nothing} (viď ukážka \ref{fig:kontrakt-bez-allocates}).

\begin{figure}[H]
    \centering
    \captionsetup{justification=centering}
\begin{lstlisting}[language=C]
//@ allocates \nothing;
void printOne ( void );
\end{lstlisting}
    \caption{Ekvivalent kontraktu bez klauzule \emph{allocates}}
    \label{fig:kontrakt-bez-allocates}
\end{figure}

\subsection{frees}

Kľúčové slovo \emph{frees} sa využíva v rámci kontraktu funkcie. Jedná sa o doplnok kľúčového slova \emph{allocates} -- je to v podstate jeho opak. V rámci tohto kľúčového slova možno špecifikovať množinu ukazateľov, ktorých pamäťové bunky sa behom funkcie uvoľnia.

Túto klauzulu si možno ukázať priamo na štandardnej funkcií jazyku C a to \emph{frees}, ktorá uvolní pamäť, na ktorú ukazuje ukazateľ v argumente funkcie.

\begin{figure}[H]
    \centering
    \captionsetup{justification=centering}
\begin{lstlisting}[language=C]
//@ frees * pointer;
void free ( void * pointer );
\end{lstlisting}
    \caption{Ukážka klauzule \emph{frees}}
    \label{fig:kontrakt-frees}
\end{figure}

Zároveň rovnako ako v prípade klauzule \emph{allocates}, ak funkcia žiadnu pamäť neuvoľňuje je možné túto klauzulu vynechať. Prípadne je možné priamo špecifikovať, že sa nič neuvoľňuje.

\begin{figure}[H]
    \centering
    \captionsetup{justification=centering}
\begin{lstlisting}[language=C]
//@ frees \nothing;
void printOne ( void );
\end{lstlisting}
    \caption{Ekvivalent kontraktu bez klauzule \emph{frees}}
    \label{fig:kontrakt-bez-frees}
\end{figure}

\subsection{predicate}

\emph{Predicate} (v preklade predikát) je z hladiska logiky logický výrazy, o ktorom možno rozhodnúť či je pravdivý, alebo nepravdivý. \cite{predikat-wiki} Jazyk \emph{ACSL} používa pre popis vstupných a výstupných podmienok logické výrazy. Tieto logické výrazy sa môžu opakovať -- môžu byť použité ako vstupné aj výstupné podmienky alebo vo viacerých funkciách. Kvôli takýmto prípadom je vhodné znižovať redundanciu, čo možno docieliť práve použitím kľúčového slova \emph{predicate}. 

Toto kľúčové slovo nám teda umožňuje špecifikovať pomenované logické výrazy obsahujúce rôzne premenné (argumenty), ktoré možno následne využívať v rámci kontraktov funkcií a tak sprehľadniť kód a znížiť jeho redundanciu. 

Túto klauzulu si ukážeme na funkcii, ktorá zmení hodnotu argumentu na absolutnú hodnotu jeho druhej odmocniny. Naviac keďže je v obore reálnych čísiel definovaná len pre nezáporné hodnoty budeme požiadavať, aby bola hodnota argumentu pred aj po vykonaní funkcie nezáporná. To docielime použitím kľúčových slov \emph{requires} a \emph{ensures}, no pre zníženie redundancie si zadefinueme predikát \emph{is\_nonnegative}, ktorý bude rozhodovať či je jeho argument nezáporné číslo (viď ukážka \ref{fig:predicate}).

\begin{figure}[H]
    \centering
    \captionsetup{justification=centering}
\begin{lstlisting}[language=C]
/*@ predicate is_nonnegative ( integer number ) = (
  @     number >= 0
  @ );
*/

/*@
  @ requires is_nonnegative ( * number );
  @
  @ ensures is_nonnegative ( * number );
*/
void squareRoot ( int * number );
\end{lstlisting}
    \caption{Ukážka klauzule \emph{predicate}}
    \label{fig:predicate}
\end{figure}

\subsection{logic}

Kľúčové slovo \emph{logic} umožňuje v rámci jazyku \emph{ACSL} špecifikovať jednoduché funkcie, ktoré možno využiť pre tvorbu predikátov a logických výrazov. Tieto funkcie nemôžu využivať premenné, ani cykly. Pre zložitejšie konštrukcie je možné použiť rekurziu.

Príkladom logickej funkcie môže byť napríklad získanie maxima v celočíselnom poli (viď ukážka \ref{fig:logic}).

\begin{figure}[H]
    \centering
    \captionsetup{justification=centering}
\begin{lstlisting}[language=C]
/*@ logic integer max_in_array ( integer * numbers, 
  @                              integer length ) = (
  @     ( length == 1 )
  @     ?
  @     numbers [ 0 ]
  @     :
  @     max ( numbers [ 0 ], 
  @           max_in_array ( numbers + 1, length - 1 ) 
  @         )
  @ );
*/
\end{lstlisting}
    \caption{Ukážka klauzule \emph{logic}}
    \label{fig:logic}
\end{figure}

\subsection{loop variant}

TODO

\subsection{loop invariant}

TODO

\subsection{loop assigns}

TODO

\subsection{behaviors}

TODO

\subsubsection{behavior}

TODO

\subsubsection{complete behaviors}

TODO

\subsubsection{disjoint behaviors}

TODO pre úplnosť

\subsection{assumes}

TODO

\subsection{ghost}

TODO

\subsection{\textbackslash result} \label{chapter:result}

TODO

\subsection{\textbackslash null}

TODO

\subsection{\textbackslash nothing} \label{chapter:nothing}

TODO

\subsection{\textbackslash valid}

TODO

\subsection{\textbackslash old} \label{chapter:old}

TODO

\subsection{\textbackslash at} 

TODO

\subsection{\textbackslash valid} 

TODO

\subsection{\textbackslash forall, \textbackslash exists}

TODO

\section{Pluginy} \label{chapter:pluginy}

TODO

\subsection{WP}

TODO

\subsection{EVA}

TODO

\subsection{RTE}

TODO


\chapter{Implementácia a overenie datovej štruktúry Union-Find}

\section{Ako sme postupovali pri implementácií?}

\section{Ako sme postupovali pri verifikácií?}

\section{Ako vyzerá výsledok overenia?}

\chapter{Testovanie datovej štruktúry Union-Find} \label{testovanie}

\section{Tvorba testov}

\subsection{Generátor}

\subsection{Fuzzer}

\section{Testovanie poškodenia pamäti} \label{symbolic_engine}

\subsection{Symbolic engine -- Klee}

\subsection{Výsledky testovania pomocou nástroja Klee}

\section{Benchmark} \label{benchmark}

\subsection{Použité nástroje}

\subsection{Výsledky benchmarku}

\include{thesis/texts/zaver.tex}

\bibliographystyle{csn690}
\bibliography{mybibliographyfile}

\appendix

\chapter{Zoznam použitých skratiek}

% \printglossaries
\begin{description}
	\item[GUI] Graphical user interface
	\item[XML] Extensible markup language
\end{description}


% % % % % % % % % % % % % % % % % % % % % % % % % % % % 
% % Tuto kapitolu z výsledné práce ODSTRAŇTE.
% % % % % % % % % % % % % % % % % % % % % % % % % % % % 
% 
% \chapter{Návod k~použití této šablony}
% 
% Tento dokument slouží jako základ pro napsání závěrečné práce na Fakultě informačních technologií ČVUT v~Praze.
% 
% \section{Výběr základu}
% 
% Vyberte si šablonu podle druhu práce (bakalářská, diplomová), jazyka (čeština, angličtina) a kódování (ASCII, \mbox{UTF-8}, \mbox{ISO-8859-2} neboli latin2 a nebo \mbox{Windows-1250}). 
% 
% V~české variantě naleznete šablony v~souborech pojmenovaných ve formátu práce\_kódování.tex. Typ může být:
% \begin{description}
% 	\item[BP] bakalářská práce,
% 	\item[DP] diplomová (magisterská) práce.
% \end{description}
% Kódování, ve kterém chcete psát, může být:
% \begin{description}
% 	\item[UTF-8] kódování Unicode,
% 	\item[ISO-8859-2] latin2,
% 	\item[Windows-1250] znaková sada 1250 Windows.
% \end{description}
% V~případě nejistoty ohledně kódování doporučujeme následující postup:
% \begin{enumerate}
% 	\item Otevřete šablony pro kódování UTF-8 v~editoru prostého textu, který chcete pro psaní práce použít -- pokud můžete texty s~diakritikou normálně přečíst, použijte tuto šablonu.
% 	\item V~opačném případě postupujte dále podle toho, jaký operační systém používáte:
% 	\begin{itemize}
% 		\item v~případě Windows použijte šablonu pro kódování \mbox{Windows-1250},
% 		\item jinak zkuste použít šablonu pro kódování \mbox{ISO-8859-2}.
% 	\end{itemize}
% \end{enumerate}
% 
% 
% 
% Více informací o~použití systému \LaTeX{} najdete např. v~\cite{wikilatex}.
% 
% \subsection{Typografie}
% 
% Při psaní dodržujte typografické konvence zvoleného jazyka. České \uv{uvozovky} zapisujte použitím příkazu \verb|\uv|, kterému v~parametru předáte text, jenž má být v~uvozovkách. Anglické otevírací uvozovky se v~\LaTeX{}u zadávají jako dva zpětné apostrofy, uzavírací uvozovky jako dva apostrofy. Často chybně uváděný symbol "{} (palce) nemá s~uvozovkami nic společného.
% 
% Dále je třeba zabránit zalomení řádky mezi některými slovy, v~češtině např. za jednopísmennými předložkami a spojkami (vyjma \uv{a}). To docílíte vložením pružné nezalomitelné mezery -- znakem \texttt{\textasciitilde}. V~tomto případě to není třeba dělat ručně, lze použít program \verb|vlna|.
% 
% Více o~typografii viz \cite{kobltypo}.
% 
% \subsection{Obrázky}
% 
% Pro umožnění vkládání obrázků je vhodné použít balíček \verb|graphicx|, samotné vložení se provede příkazem \verb|\includegraphics|. Takto je možné vkládat obrázky ve formátu PDF, PNG a JPEG jestliže používáte pdf\LaTeX{} nebo ve formátu EPS jestliže používáte \LaTeX{}. Doporučujeme preferovat vektorové obrázky před rastrovými (vyjma fotografií).
% 
% \subsubsection{Získání vhodného formátu}
% 
% Pro získání vektorových formátů PDF nebo EPS z~jiných lze použít některý z~vektorových grafických editorů. Pro převod rastrového obrázku na vektorový lze použít rasterizaci, kterou mnohé editory zvládají (např. Inkscape). Pro konverze lze použít též nástroje pro dávkové zpracování běžně dodávané s~\LaTeX{}em, např. \verb|epstopdf|.
% 
% \subsubsection{Plovoucí prostředí}
% 
% Příkazem \verb|\includegraphics| lze obrázky vkládat přímo, doporučujeme však použít plovoucí prostředí, konkrétně \verb|figure|. Například obrázek \ref{fig:float} byl vložen tímto způsobem. Vůbec přitom nevadí, když je obrázek umístěn jinde, než bylo původně zamýšleno -- je tomu tak hlavně kvůli dodržení typografických konvencí. Namísto vynucování konkrétní pozice obrázku doporučujeme používat odkazování z~textu (dvojice příkazů \verb|\label| a \verb|\ref|).
% 
% \begin{figure}\centering
% 	\includegraphics[width=0.5\textwidth, angle=30]{cvut-logo-bw}
% 	\caption[Příklad obrázku]{Ukázkový obrázek v~plovoucím prostředí}\label{fig:float}
% \end{figure}
% 
% \subsubsection{Verze obrázků}
% 
% % Gnuplot BW i barevně
% Může se hodit mít více verzí stejného obrázku, např. pro barevný či černobílý tisk a nebo pro prezentaci. S~pomocí některých nástrojů na generování grafiky je to snadné.
% 
% Máte-li například graf vytvořený v programu Gnuplot, můžete jeho černobílou variantu (viz obr. \ref{fig:gnuplot-bw}) vytvořit parametrem \verb|monochrome dashed| příkazu \verb|set term|. Barevnou variantu (viz obr. \ref{fig:gnuplot-col}) vhodnou na prezentace lze vytvořit parametrem \verb|colour solid|.
% 
% \begin{figure}\centering
% 	\includegraphics{gnuplot-bw}
% 	\caption{Černobílá varianta obrázku generovaného programem Gnuplot}\label{fig:gnuplot-bw}
% \end{figure}
% 
% \begin{figure}\centering
% 	\includegraphics{gnuplot-col}
% 	\caption{Barevná varianta obrázku generovaného programem Gnuplot}\label{fig:gnuplot-col}
% \end{figure}
% 
% 
% \subsection{Tabulky}
% 
% Tabulky lze zadávat různě, např. v~prostředí \verb|tabular|, avšak pro jejich vkládání platí to samé, co pro obrázky -- použijte plovoucí prostředí, v~tomto případě \verb|table|. Například tabulka \ref{tab:matematika} byla vložena tímto způsobem.
% 
% \begin{table}\centering
% 	\caption[Příklad tabulky]{Zadávání matematiky}\label{tab:matematika}
% 	\begin{tabular}{|l|l|c|c|}\hline
% 		Typ		& Prostředí		& \LaTeX{}ovská zkratka	& \TeX{}ovská zkratka	\tabularnewline \hline \hline
% 		Text		& \verb|math|		& \verb|\(...\)|	& \verb|$...$|		\tabularnewline \hline
% 		Displayed	& \verb|displaymath|	& \verb|\[...\]|	& \verb|$$...$$|	\tabularnewline \hline
% 	\end{tabular}
% \end{table}
% 
% % % % % % % % % % % % % % % % % % % % % % % % % % % % 

\chapter{Obsah priloženého CD}

%upravte podle skutecnosti

\begin{figure}
	\dirtree{%
		.1 readme.txt\DTcomment{stručný popis obsahu CD}.
		.1 exe\DTcomment{adresár so spustiteľnou formou implementácie}.
		.1 src.
		.2 impl\DTcomment{zdrojové kódy implementácie}.
		.2 thesis\DTcomment{zdrojová forma práce vo formáte \LaTeX{}}.
		.1 text\DTcomment{text práce}.
		.2 thesis.pdf\DTcomment{text práce vo formáte PDF}.
		.2 thesis.ps\DTcomment{text práce vo formáte PS}.
	}
\end{figure}

\end{document}
